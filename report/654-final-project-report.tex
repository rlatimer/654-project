% Options for packages loaded elsewhere
\PassOptionsToPackage{unicode}{hyperref}
\PassOptionsToPackage{hyphens}{url}
%
\documentclass[
  11pt,
]{article}
\usepackage{lmodern}
\usepackage{amsmath}
\usepackage{ifxetex,ifluatex}
\ifnum 0\ifxetex 1\fi\ifluatex 1\fi=0 % if pdftex
  \usepackage[T1]{fontenc}
  \usepackage[utf8]{inputenc}
  \usepackage{textcomp} % provide euro and other symbols
  \usepackage{amssymb}
\else % if luatex or xetex
  \usepackage{unicode-math}
  \defaultfontfeatures{Scale=MatchLowercase}
  \defaultfontfeatures[\rmfamily]{Ligatures=TeX,Scale=1}
\fi
% Use upquote if available, for straight quotes in verbatim environments
\IfFileExists{upquote.sty}{\usepackage{upquote}}{}
\IfFileExists{microtype.sty}{% use microtype if available
  \usepackage[]{microtype}
  \UseMicrotypeSet[protrusion]{basicmath} % disable protrusion for tt fonts
}{}
\makeatletter
\@ifundefined{KOMAClassName}{% if non-KOMA class
  \IfFileExists{parskip.sty}{%
    \usepackage{parskip}
  }{% else
    \setlength{\parindent}{0pt}
    \setlength{\parskip}{6pt plus 2pt minus 1pt}}
}{% if KOMA class
  \KOMAoptions{parskip=half}}
\makeatother
\usepackage{xcolor}
\IfFileExists{xurl.sty}{\usepackage{xurl}}{} % add URL line breaks if available
\IfFileExists{bookmark.sty}{\usepackage{bookmark}}{\usepackage{hyperref}}
\hypersetup{
  pdftitle={Predicting Length of Stay for Shelter Dogs},
  pdfauthor={Rachael Latimer},
  hidelinks,
  pdfcreator={LaTeX via pandoc}}
\urlstyle{same} % disable monospaced font for URLs
\usepackage[margin=1in]{geometry}
\usepackage{color}
\usepackage{fancyvrb}
\newcommand{\VerbBar}{|}
\newcommand{\VERB}{\Verb[commandchars=\\\{\}]}
\DefineVerbatimEnvironment{Highlighting}{Verbatim}{commandchars=\\\{\}}
% Add ',fontsize=\small' for more characters per line
\usepackage{framed}
\definecolor{shadecolor}{RGB}{248,248,248}
\newenvironment{Shaded}{\begin{snugshade}}{\end{snugshade}}
\newcommand{\AlertTok}[1]{\textcolor[rgb]{0.94,0.16,0.16}{#1}}
\newcommand{\AnnotationTok}[1]{\textcolor[rgb]{0.56,0.35,0.01}{\textbf{\textit{#1}}}}
\newcommand{\AttributeTok}[1]{\textcolor[rgb]{0.77,0.63,0.00}{#1}}
\newcommand{\BaseNTok}[1]{\textcolor[rgb]{0.00,0.00,0.81}{#1}}
\newcommand{\BuiltInTok}[1]{#1}
\newcommand{\CharTok}[1]{\textcolor[rgb]{0.31,0.60,0.02}{#1}}
\newcommand{\CommentTok}[1]{\textcolor[rgb]{0.56,0.35,0.01}{\textit{#1}}}
\newcommand{\CommentVarTok}[1]{\textcolor[rgb]{0.56,0.35,0.01}{\textbf{\textit{#1}}}}
\newcommand{\ConstantTok}[1]{\textcolor[rgb]{0.00,0.00,0.00}{#1}}
\newcommand{\ControlFlowTok}[1]{\textcolor[rgb]{0.13,0.29,0.53}{\textbf{#1}}}
\newcommand{\DataTypeTok}[1]{\textcolor[rgb]{0.13,0.29,0.53}{#1}}
\newcommand{\DecValTok}[1]{\textcolor[rgb]{0.00,0.00,0.81}{#1}}
\newcommand{\DocumentationTok}[1]{\textcolor[rgb]{0.56,0.35,0.01}{\textbf{\textit{#1}}}}
\newcommand{\ErrorTok}[1]{\textcolor[rgb]{0.64,0.00,0.00}{\textbf{#1}}}
\newcommand{\ExtensionTok}[1]{#1}
\newcommand{\FloatTok}[1]{\textcolor[rgb]{0.00,0.00,0.81}{#1}}
\newcommand{\FunctionTok}[1]{\textcolor[rgb]{0.00,0.00,0.00}{#1}}
\newcommand{\ImportTok}[1]{#1}
\newcommand{\InformationTok}[1]{\textcolor[rgb]{0.56,0.35,0.01}{\textbf{\textit{#1}}}}
\newcommand{\KeywordTok}[1]{\textcolor[rgb]{0.13,0.29,0.53}{\textbf{#1}}}
\newcommand{\NormalTok}[1]{#1}
\newcommand{\OperatorTok}[1]{\textcolor[rgb]{0.81,0.36,0.00}{\textbf{#1}}}
\newcommand{\OtherTok}[1]{\textcolor[rgb]{0.56,0.35,0.01}{#1}}
\newcommand{\PreprocessorTok}[1]{\textcolor[rgb]{0.56,0.35,0.01}{\textit{#1}}}
\newcommand{\RegionMarkerTok}[1]{#1}
\newcommand{\SpecialCharTok}[1]{\textcolor[rgb]{0.00,0.00,0.00}{#1}}
\newcommand{\SpecialStringTok}[1]{\textcolor[rgb]{0.31,0.60,0.02}{#1}}
\newcommand{\StringTok}[1]{\textcolor[rgb]{0.31,0.60,0.02}{#1}}
\newcommand{\VariableTok}[1]{\textcolor[rgb]{0.00,0.00,0.00}{#1}}
\newcommand{\VerbatimStringTok}[1]{\textcolor[rgb]{0.31,0.60,0.02}{#1}}
\newcommand{\WarningTok}[1]{\textcolor[rgb]{0.56,0.35,0.01}{\textbf{\textit{#1}}}}
\usepackage{graphicx}
\makeatletter
\def\maxwidth{\ifdim\Gin@nat@width>\linewidth\linewidth\else\Gin@nat@width\fi}
\def\maxheight{\ifdim\Gin@nat@height>\textheight\textheight\else\Gin@nat@height\fi}
\makeatother
% Scale images if necessary, so that they will not overflow the page
% margins by default, and it is still possible to overwrite the defaults
% using explicit options in \includegraphics[width, height, ...]{}
\setkeys{Gin}{width=\maxwidth,height=\maxheight,keepaspectratio}
% Set default figure placement to htbp
\makeatletter
\def\fps@figure{htbp}
\makeatother
\setlength{\emergencystretch}{3em} % prevent overfull lines
\providecommand{\tightlist}{%
  \setlength{\itemsep}{0pt}\setlength{\parskip}{0pt}}
\setcounter{secnumdepth}{-\maxdimen} % remove section numbering
\usepackage{booktabs}
\usepackage{longtable}
\usepackage{array}
\usepackage{multirow}
\usepackage{wrapfig}
\usepackage{float}
\usepackage{colortbl}
\usepackage{pdflscape}
\usepackage{tabu}
\usepackage{threeparttable}
\usepackage{threeparttablex}
\usepackage[normalem]{ulem}
\usepackage{makecell}
\usepackage{xcolor}
\usepackage{amsmath}
\usepackage{caption}
\ifluatex
  \usepackage{selnolig}  % disable illegal ligatures
\fi

\title{Predicting Length of Stay for Shelter Dogs}
\author{Rachael Latimer}
\date{}

\begin{document}
\maketitle

~~~~As people stayed home for most of the early days of the coronavirus
pandemic, the demand for goods and services increased. Some of the
increased demand was not surprising and likely expected; items such as
home workout equipment, trampolines, and lumber. However, some of the
demand took the industry by surprise: yeast for baking, and pets. In
fact, the interest in pet adoption increased so much that shelters were
regularly reporting empty kennels and sifting through dozens of adoption
applications for a single puppy.\\
\hspace*{0.333em}\hspace*{0.333em}\hspace*{0.333em}\hspace*{0.333em}Unfortunately,
as vaccines were rolled out and people began returning to work and
school, shelters and foster groups filled up with animals that were no
longer compatible with people's lifestyles. The decrease in demand for
dogs means that people can be more selective in the kind of dog they
adopt. However, the information provided by animal shelters and rescue
groups are often based on a short period of time with the animal and the
animal's appearance. One shelter's Border collie mix might be another's
spaniel or shepherd mix. This best guess breed identification can have
significant impacts on a dog's future and could be the difference
between adoption and euthanasia. Objectively identifying the impact of a
dog's listed breed on the length of stay in an animal shelter could
provide shelters with the information needed to shift away from listing
a dog's breed as the primary information for potential adopter and
toward a more holistic evaluation of a dog's temperament and future
needs.

\begin{Shaded}
\begin{Highlighting}[]
\FunctionTok{summary}\NormalTok{(cars)}
\end{Highlighting}
\end{Shaded}

\begin{verbatim}
##      speed           dist       
##  Min.   : 4.0   Min.   :  2.00  
##  1st Qu.:12.0   1st Qu.: 26.00  
##  Median :15.0   Median : 36.00  
##  Mean   :15.4   Mean   : 42.98  
##  3rd Qu.:19.0   3rd Qu.: 56.00  
##  Max.   :25.0   Max.   :120.00
\end{verbatim}

\hypertarget{data}{%
\subsection{Data}\label{data}}

~~~~The Austin Animal Center in Texas is the largest no-kill shelter in
the US. The shelter maintains data on the intake and outcomes of animals
beginning from October 2013 to present. This data set was obtained from
kaggle:
(\url{https://www.kaggle.com/aaronschlegel/austin-animal-center-shelter-intakes-and-outcomes?select=aac_intakes_outcomes.csv}).
It was originally provided by the Austin Animal Center in Austin, Texas.
The data include information about the intake and outcome of the animal,
and details on the type and condition of the animal. A brief examination
of the data revealed that the animal shelter takes in animals in
addition to typical domestic pets (cats and dogs). For the purposes of
this study, the following types of animals were excluded: cats, birds,
and animals that were classified as other, including rabbits, bats,
snakes, raccoons, ferrets, reptiles, and other wild animals that live in
close proximity to humans. Additionally, dog breeds with sample sizes
less than 20 were excluded as this small sample made it difficult to
accurately model the length of stay for the breed. The final data set
included variables of the animal (breed, age on intake, sex, condition
of the animal), circumstances of the animal arriving at the shelter
(type of intake, month of intake), and specifics of the outcome of the
animal (outcome, month of outcome, time spent in the shelter, measured
in days).\\
\hspace*{0.333em}\hspace*{0.333em}\hspace*{0.333em}\hspace*{0.333em}Initial
data visualization was performed to understand the data available. This
included visualizing the number of animals taken into the shelter each
month (Figure 1) and further exploring the number of each type of animal
taken in each month (Table 1).

\begin{center}\includegraphics{654-final-project-report_files/figure-latex/fig1-1} \end{center}

\textbf{Figure 1.} \emph{Shelter Animals Taken in Each Month}

\textbf{Table 1.} \emph{Type of Shelter Animal Taken in Each Month}

\captionsetup[table]{labelformat=empty,skip=1pt}
\begin{longtable}{rrrrr}
\toprule
 & \multicolumn{4}{c}{Animal Type} \\ 
 \cmidrule(lr){2-5}
Intake Month & Bird & Cat & Dog & Other \\ 
\midrule
1 & 19 & 1529 & 4181 & 253 \\ 
2 & 61 & 1320 & 3970 & 326 \\ 
3 & 26 & 1600 & 4096 & 891 \\ 
4 & 35 & 2330 & 3353 & 394 \\ 
5 & 38 & 3741 & 3867 & 320 \\ 
6 & 24 & 3519 & 3556 & 360 \\ 
7 & 25 & 2893 & 3312 & 298 \\ 
8 & 22 & 2819 & 3271 & 386 \\ 
9 & 23 & 2756 & 3420 & 290 \\ 
10 & 29 & 3003 & 4215 & 405 \\ 
11 & 19 & 2285 & 4038 & 255 \\ 
12 & 18 & 1744 & 4087 & 250 \\ 
 \bottomrule
\end{longtable}

It is also of interest to explore the relationship between the average
length of stay in the shelter of each breed with the frequency of that
breed present in the shelter. Table 2 provides the average length of
stay in the shelter for the 25 most frequently taken in dog breeds.

\textbf{Table 2.} \emph{Average Length of Stay (in days) for Most Common
Dog Breeds in Shelter} \begingroup\fontsize{11}{13}\selectfont

\begin{tabular}{l|r|r}
\hline
Breed & N & Avg. Length of Stay (days)\\
\hline
Pit Bull & 6865 & 28.028652\\
\hline
Labrador Retriever & 6260 & 16.906381\\
\hline
Chihuahua Shorthair & 5726 & 11.841401\\
\hline
German Shepherd & 2612 & 12.771461\\
\hline
Australian Cattle Dog & 1442 & 19.901094\\
\hline
Dachshund & 1239 & 8.671506\\
\hline
Boxer & 948 & 19.438599\\
\hline
Border Collie & 928 & 12.298963\\
\hline
Miniature Poodle & 841 & 7.600594\\
\hline
Beagle & 642 & 12.259929\\
\hline
Siberian Husky & 641 & 8.506920\\
\hline
Catahoula & 638 & 23.371538\\
\hline
Australian Shepherd & 624 & 10.586646\\
\hline
Jack Russell Terrier & 623 & 10.260468\\
\hline
Yorkshire Terrier & 621 & 7.316022\\
\hline
Rat Terrier & 598 & 10.200137\\
\hline
Miniature Schnauzer & 584 & 5.601945\\
\hline
Great Pyrenees & 516 & 10.899884\\
\hline
Rottweiler & 493 & 18.182440\\
\hline
Shih Tzu & 463 & 4.929401\\
\hline
Pointer & 448 & 22.045905\\
\hline
Chihuahua Longhair & 446 & 7.845548\\
\hline
Cairn Terrier & 436 & 8.984851\\
\hline
Staffordshire & 418 & 28.627440\\
\hline
American Bulldog & 381 & 39.187939\\
\hline
\end{tabular}
\endgroup{}

\hypertarget{models}{%
\subsection{Models}\label{models}}

\begin{verbatim}
##   intercept
## 1      TRUE
\end{verbatim}

\begin{verbatim}
## $everything
##    user  system elapsed 
##   36.09    0.50   12.46 
## 
## $final
##    user  system elapsed 
##    3.53    0.02    1.04 
## 
## $prediction
## [1] NA NA NA
\end{verbatim}

~~~~In order to predict the length of stay of shelter dogs, three types
of modeling approaches were explored: linear regression, linear
regression with ridge penalty, and bagged trees. These models were
chosen for their increasing complexity to determine the extent to which
the increasing complexity added value to or impacted the predictions and
importance of variables used in the predictions. All models were fit
with 10-fold cross validation for comparison purposes. The performances
of the models will be compared using the values of R-squared, MAE, and
RMSE.

\hypertarget{results}{%
\subsection{Results}\label{results}}

~~~~The linear regression without regularization produced an RMSE of
37.66, and MAE of 16.77 and an r-squared of .11. The linear regression
with ridge penalty produced an RMSE of 37.66, MAE of 16.57, and
r-squared of .11. These results are very similar to those of the linear
regression without regularization. The model also revealed the top ten
predictors of length of stay for shelter dogs, which included eight
specific breeds (Table 3).

\textbf{Table 3.} \emph{Top Ten Predictors of Shelter Length of Stay
from Linear Regression with Ridge Penalty Model}

\begin{verbatim}
##                                      [,1]
## na_ind_outcome_type             -29.54081
## breed_American.Bulldog           24.37111
## breed_Bulldog                    20.68927
## (Intercept)                      19.10515
## breed_Collie.Smooth              18.35328
## breed_Tibetan.Spaniel           -18.22182
## breed_English.Coonhound          17.58684
## breed_American.Pit.Bull.Terrier  17.40489
## breed_Silky.Terrier             -16.77498
## breed_Flat.Coat.Retriever        14.29413
\end{verbatim}

~~~The bagged tree model had a similar RMSE as the linear regression
without regularization and the linear regression with ridge penalty
(34.45), a lower MAE of 12.36, and a higher r-squared at .26. Due to the
similar RMSE and MAE of all three models, I would choose the bagged tree
model due to the higher r-squared value. The three models are compared
in Table 4.

\textbf{Table 4.} \emph{Model Comparison}

\begin{verbatim}
##                                  Model     RMSE      MAE       Rsq
## 1                    Linear Regression 37.65603 16.76878 0.1077589
## 2 Linear Regression with Ridge Penalty 37.66347 16.57493 0.1068479
## 3                   Bagged Trees Model 34.45266 12.36240 0.2590010
\end{verbatim}

\hypertarget{discussion}{%
\subsection{Discussion}\label{discussion}}

~~~~As Table 3 illustrated, the breed of the dog is highly predictive of
the amount of time a dog spends in an animal shelter. Although there are
a number of large breed dogs, there is not an obvious category of dog
that is more predictive of the length of stay. For example, there are
representatives of a variety of AKC dog groups: the non-sporting group
(Bulldog, Tibetan Spaniel), the herding group (Collie), hound
(Coonhound); sizes: the Terrier and the Tibetan Spaniel being small
dogs, and the Retriever and Bulldog being larger dogs, and temperament.
This suggests there may be some prejudice or unfamiliarity of some
breeds.

\hypertarget{conclusion}{%
\subsection{Conclusion}\label{conclusion}}

~~~~This wide variety of dog breeds in the top ten category is
surprising. I would have thoughts that the larger dogs and bully-type
dogs would occupy the top spots. I was not, however, surprised that
breed was a top predictor of the length of stay in the shelter. The
breed of the dog is often the best predictor of temperament of a dog,
however if the breed listed is only based on a dog's appearance and not
genetics or the dog's history, then breed is less powerful of a
predictor. Shelters may find they have better success matching shelter
dogs with their forever families if they provide a more thorough
behavioral evaluation of the dog and assessment of the dog's future
needs to potential adopters.

\end{document}
